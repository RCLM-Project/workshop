\documentclass[10pt,twoside]{article}
\usepackage{ssrq_de}  % here's all the magic...
\usepackage{hyperref}
\begin{document}
% Variablen für Kopfzeile
\def\volume{XIV. Abteilung: Die Rechtsquellen des Kantons St. Gallen, Dritter Teil: Die Landschaften und Landstädte, Band 4: Die Rechtsquellen der
     Region Werdenberg: Grafschaft Werdenberg und Herrschaft Wartau, Freiherrschaft Sax-Forstegg und Herrschaft Hohensax-Gams von Sibylle Malamud, 2019.}
\def\volid{\url{https://www.ssrq-sds-fds.ch/online/tei/SG/SSRQ_SG_III_4_253.xml}}
\linenumbers				% Zeilennummern
\normalsize					% normale Grösse, d.\,h. Quellentext (source) ist 10pt
\thispagestyle{firstpage}
\sloppy
\setcounter{subsection}{252}  % Stücknummer - 1
\article{Gedruckter Steckbrief der Kanzlei Bern an die eidgenössischen Stände über die Verbannung von Johannes Moser aus Gams (Beutelschneiderei)}
\fussy
\dating{1787 Juni 9. Kanzlei Bern}
\begin{introlist}

         \item {Der \term{key001017}{Steckbrief} wurde einem allgemeinen, an die \persname{org000531}{Eidgenossen} adressierten Schreiben vom 9. Juni 1787 aus der Stadt \placename{loc000157}{Bern} beigelegt. Dieses Exemplar ist an \persname{org004001}{Landammann und Rat von Unterwalden} verschickt worden und liegt heute im
               Staatsarchiv Nidwalden. \persname{org004002}{Schultheiss und Rat der Stadt
                  Bern} ersuchen mit dem Steckbrief die eidgenössischen Orte, dem Gesuchten
               in ihrem \term{key003544}{Land} keinen Aufenthalt zu gewähren, denn sie
               haben \persname{per010723a}{Johannes Moser} von \placename{loc000211}{Gams}, \textup{der herrschafft \placename{loc000440}{Werdenberg}, einen \term{lem016388.02}{beütelschneider},
                  [...] für lebenslang \term{lem012781}{aus unseren landen} und aus
                  gesamt loblichen
                  \placename{loc000153}{Eidgenoßschafft} verwiesen}
                     (StANW C 1025/8:87). Fälschlicherweise verortet
               Bern Gams in der Landvogtei Werdenberg anstatt in der Herrschaft \placename{loc006040}{Hohensax-Gams} (auch im Steckbrief ebenso falsch als
                  \textup{\placename{loc007494}{Württemberg}} gedruckt, siehe Fussnote 1). 


}
         \item {Weitere Steckbriefe siehe StASG AA 3 B 6, 16.11.1735; LAGL AG III.25, Bündel 111, 26.01.1741; StANW C 1025/6:47; StABE DQ 1167; StALU PA 211/377;
            PA Hilty S 006/083; S 006/099; S 006/101.


}
      
\end{introlist}


\begin{source}

               
\noindent \term{lem016380.01}{Signalement}

\vspace{1.5mm}
                  
\noindent \persname{per010723a}{Johannes Moser}, von \placename{loc000211}{Gams} aus dem \placename{loc007494.03}{Würtembergischen\,[!]}\leavevmode\ednote[1]{Im Begleitbrief zum Steckbrief heisst es Johannes Moser von Gams aus Werdenberg, weshalb es sich hier um einen Verschrieb handelt. Zudem verortet Bern Gams fälschlicherweise in der Landvogtei Werdenberg anstatt in der Herrschaft \placename{loc006040}{Hohensax-Gams} (StANW C 1025/8:87, siehe auch Kommentar 1).},
                     bey 26 \term{lem000732}{Jahr} alt, bey 5 Schuh
                           2½ Zoll hoch,
                     \placename{loc000281}{französisches}
                     \term{lem000196}{Mäs}, etwas blassen \term{lem004441}{magern}
                     \term{lem004416}{Angesichts}, hat braune \term{lem003827}{Augen},
                     an dem einten \term{lem016382.01}{Backen}
                     eine \term{lem016383.01}{Narbe}, schöne weiße \term{lem016384.02}{Zähne}, \term{lem000979}{schwarze}
                     \term{lem004664}{Haare}, \term{lem016381.02}{Augsbraunen} und
                     \term{lem005660}{Bart}; trug ein braunes \term{lem003502}{Kleid}
                     und Weste mit stählernen \term{lem016385.02}{Knöpfen}, \term{lem000388}{Hosen} von \term{lem000979}{schwarzem}
                     \term{lem016386.01}{Manchester} und \term{lem016387.01}{Stiefel};
                     redet \term{lem002050}{Teutsch}
                     und \term{lem004992}{Französisch}.


                  Geben, den 9. Brachmonats 1787.

 Canzley
                     \placename{loc000157}{Bern}.


               
               
            
\end{source}



\manudesc{\textbf{Druck:} StANW C 1025/8:87; (Einzelblatt); Papier, 22.0\,×\,36.0\,cm.}





\printnotes*
\end{document}
